\documentclass{article}[12 pt]
\usepackage{amssymb}
\usepackage{amsthm}
\usepackage{amsmath}
\usepackage{appendix}
\usepackage{array}
\usepackage{geometry}
\usepackage{enumitem}
\usepackage{graphicx}
\usepackage{subfig}
\usepackage{caption}
\usepackage{url}
\usepackage{float}
\usepackage{pdfpages}
\usepackage{shortvrb}
\usepackage{mathtools}
\usepackage{multirow}
\usepackage{hyperref}

\def\BibTeX{{\rm B\kern-.05em{\sc i\kern-.025em b}\kern-.08em
		T\kern-.1667em\lower.7ex\hbox{E}\kern-.125emX}}

\graphicspath{{"K:/University of Florida/Classes/2018_08_Woodard_Image_Processing/Homework/HW01/"}}
\geometry{margin=1 in}

\newcommand{\smallvskip}{\vspace{5 pt}}
\newcommand{\medvskip}{\vspace{30 pt}}
\newcommand{\bigvskip}{\vspace{100 pt}}
\newcommand{\tR}{\mathtt{R}}




\begin{document}
	
\begin{center}
	\textbf{\Large Connor McCurley} \\
	EEE 6512 \qquad \textbf{\large Homework 1 Due August 23, 2018} \qquad Fall 2018 
\end{center}


This document demonstrates the interweaving of two images consisting of only the odd and even rows of a full image, respectively.  Additionally, the even rows were corrupted.  As demonstrated in figure \ref{fig:images}, the original image was reconstruced by first un-corrupting the even rows by adding the means of their respective odd rows.  The odd row and uneven row images were then interweaved to obtain the original image. \\
 
 \noindent Accompanying code is provided in $mccurleyHW01.mat$

\begin{center}
	\begin{figure}[h]
		\centering
		\includegraphics[width=\textwidth]{"hw01_images"}
		\caption{Top Left: Odd Rows of original image Top Right: Corrupted, even rows of original image. Bottom Left: Original, corrupted image Bottom Right: Original, un-corrupted image}
		\label{fig:images}
	\end{figure}
\end{center}









\end{document}
